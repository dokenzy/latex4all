\hypersetup{
	pdfauthor={dokenzy, nova de hi},
    pdftitle={모두를 위한 LaTeX},
    pdfsubject={LaTeX을 시작하려는 사람들을 위한 성의없는 설명서},
    pdfkeywords={LaTeX, 매뉴얼, 너무 성의없다},
    colorlinks= true,
    linkcolor=blue,
    urlcolor=blue,
    citecolor=blue,
    anchorcolor=blue
}
\newcommand{\inlinecode}[1]{\raisebox{0.25ex}{#1}}
\newcommand{\google}{구글}
\newcommand{\texworks}{TeXworks}

\setlength\parskip{1.0em}
\setlength\parindent{0pt}

\tcbset{listing engine=minted}

\newtcblisting[auto counter]{pyglist}
	{fonttitle=\sffamily\small,title=Code~\thetcbcounter,breakable,listing only,
	 minted options={tabsize=4,fontsize=\small},
	 colframe=Goldenrod,
	}
\newtcblisting{showresult}{title=결과,fonttitle=\sffamily\small,text only,colframe=Goldenrod,colback=Cornsilk1}


\usepackage{mdframed}

% 그림 캡션 모양
\captiondelim{ } % "그림:" 에서 ":"을 없앰
\captionnamefont{\footnotesize\bfseries\sffamily} % "그림"을 굵게
\captiontitlefont{\footnotesize\sffamily} % 본문은 보통 굵기로

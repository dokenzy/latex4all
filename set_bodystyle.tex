\hypersetup{
	pdfauthor={dokenzy},
    pdftitle={모두를 위한 LaTeX},
    pdfsubject={LaTeX을 시작하려는 사람들을 위한 성의없는 설명서},
    pdfkeywords={LaTeX, 매뉴얼, 너무 성의없다},
    colorlinks= true,
    linkcolor=blue,
    urlcolor=blue,
    citecolor=blue,
    anchorcolor=blue
}
\newcommand{\inlinecode}[1]{\raisebox{0.25ex}{#1}}
\newcommand{\google}{구글}
\newcommand{\texworks}{TeXworks}

\setlength\parskip{1.0em}
\setlength\parindent{0pt}

% 예제결과 박스
\newmdenv[%
	nobreak=true,
	roundcorner=5pt,
	linewidth=0.3pt,
	linecolor=Goldenrod,
	subtitlebelowline=true,
	subtitleaboveline=true,
	backgroundcolor=white, 
	frametitle=결과,
	frametitlefont=\ttfamily\small,
	frametitlerule=true, 
	frametitlerulewidth=0.5pt,
	frametitlebackgroundcolor=Gold
]{showresult}

% 예제코드 박스
\mdtheorem[%
	nobreak=true,
	linewidth=0.3pt ,%
	innerleftmargin=5pt,
	innerrightmargin=5pt,
	innertopmargin=15pt,
	innerbottommargin=3pt,
	leftmargin = 40 ,%
	rightmargin = 40 ,%
	leftmargin=0em,
	rightmargin=0em,
	linecolor=Goldenrod,
	roundcorner=5pt,
	backgroundcolor=Cornsilk1,
	frametitlerule=true,%
	frametitlebackgroundcolor=Gold,
	frametitlefont=\sffamily\small,
	frametitleaboveskip=2pt,
	frametitlebelowskip=2pt]
{mdcodebox}{Code}

\BeforeBeginEnvironment{pyglist}{\begin{mdcodebox}}
\AfterEndEnvironment{pyglist}{\end{mdcodebox}}

% 그림 캡션 모양
\captiondelim{ } % "그림:" 에서 ":"을 없앰
\captionnamefont{\footnotesize\bfseries\sffamily} % "그림"을 굵게
\captiontitlefont{\footnotesize\sffamily} % 본문은 보통 굵기로
